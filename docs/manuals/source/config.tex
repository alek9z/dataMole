%**************************************************************
% file contenente le impostazioni della tesi
%**************************************************************

%**************************************************************
% Impostazioni di impaginazione
% see: http://wwwcdf.pd.infn.it/AppuntiLinux/a2547.htm
%**************************************************************

\setlength{\parindent}{0pt}   % larghezza rientro della prima riga
\setlength{\parskip}{0pt}   % distanza tra i paragrafi


%**************************************************************
% Impostazioni di biblatex
%**************************************************************
\bibliography{bibliography} % database di biblatex 

%\defbibheading{bibliography} {
%    \cleardoublepage
%    \phantomsection
%    \addcontentsline{toc}{chapter}{\bibname}
%    \chapter*{\bibname\markboth{\bibname}{\bibname}}
%}

\setlength\bibitemsep{1.5\itemsep} % spazio tra entry

%\DeclareBibliographyCategory{opere}
%\DeclareBibliographyCategory{web}

%**************************************************************
% Impostazioni di caption
%**************************************************************
%\captionsetup{
%    tableposition=top,
%    figureposition=bottom,
%    font=small,
%    format=hang,
%    labelfont=bf
%}

%**************************************************************
% Impostazioni di glossaries
%**************************************************************
%\input{special_sections/glossary} % database di termini
%\makeglossaries


%**************************************************************
% Impostazioni di graphicx
%**************************************************************

%\setcounter{secnumdepth}{3}

\setcounter{secnumdepth}{4}

\titleformat{\paragraph} {\normalfont\normalsize\bfseries}{\theparagraph}{1em}{}
\titlespacing*{\paragraph} {0pt}{3.25ex plus 1ex minus .2ex}{1.5ex plus .2ex}

%\newcommand{\lineparagraph}[1]{\paragraph{#1}\mbox{}\\}
\newcommand{\lineparagraph}[1]{\paragraph{#1}}

% TABELLE STRETCH
%\renewcommand{\arraystretch}{1.5}
%\setlength{\arrayrulewidth}{1.2pt}


% table colors

\definecolor{myblue}{rgb}{0.0, 0.47, 0.75}
\definecolor{mylightgray}{RGB}{242, 242, 242}
\definecolor{OliveGreen}{rgb}{0.55, 0.71, 0.0}
%\definecolor{tableCol}{rgb}{0.88,1,1}
\colorlet{tableCol}{mylightgray}
%\definecolor{tableHead}{HTML}{EFEFEF}
\colorlet{tableHead}{myblue}


%**************************************************************
% Impostazioni di hyperref
%**************************************************************
\hypersetup{
    %hyperfootnotes=false,
    %pdfpagelabels,
    %draft,	% = elimina tutti i link (utile per stampe in bianco e nero)
    colorlinks=true,
    linktocpage=true,
    pdfstartpage=1,
    pdfstartview=,
    % decommenta la riga seguente per avere link in nero (per esempio per la stampa in bianco e nero)
    %colorlinks=false, linktocpage=false, pdfborder={0 0 0}, pdfstartpage=1, pdfstartview=FitV,
    breaklinks=true,
    pdfpagemode=UseNone,
    pageanchor=true,
    pdfpagemode=UseOutlines,
    plainpages=false,
    bookmarksnumbered,
    bookmarksopen=true,
    bookmarksopenlevel=1,
    hypertexnames=true,
    pdfhighlight=/O,
    %nesting=true,
    %frenchlinks,
    urlcolor=webbrown,
    linkcolor=myblue,
    citecolor=webgreen
}


%\makeatletter
%%Create an OutputIntent in order to correctly specify colours
%\immediate\pdfobj stream attr{/N 3} file{AdobeRGB1998.icc}
%\edef\iccobj{\the\pdflastobj}
%\pdfcatalog{%
%/OutputIntents [
%<<
%/Type /OutputIntent
%/S/GTS_PDFA1
%/DestOutputProfile \iccobj\space 0 R
%/OutputConditionIdentifier (AdobeRGB1998)
%/Info (AdobeRGB1998)
%>>
%]
%}
%\makeatother

%**************************************************************
% Impostazioni di itemize
%**************************************************************
%\renewcommand{\labelitemi}{$\ast$}

%\renewcommand{\labelitemi}{$\bullet$}
%\renewcommand{\labelitemii}{$\cdot$}
%\renewcommand{\labelitemiii}{$\diamond$}
%\renewcommand{\labelitemiv}{$\ast$}


%**************************************************************
% Impostazioni di listings
%**************************************************************
\lstset{
    language=[LaTeX]Tex,%C++,
    keywordstyle=\color{RoyalBlue}, %\bfseries,
    basicstyle=\small\ttfamily,
    %identifierstyle=\color{NavyBlue},
    commentstyle=\color{Green}\ttfamily,
    stringstyle=\rmfamily,
    numbers=none, %left,%
    numberstyle=\scriptsize, %\tiny
    stepnumber=5,
    numbersep=8pt,
    showstringspaces=false,
    breaklines=true,
    frameround=ftff,
    frame=single
}

\lstdefinelanguage{JavaScript}{
    keywords={typeof, new, true, false, catch, function, return, null, catch, switch, var, if, in, while, do, else, case, break},
    keywordstyle=\color{blue}\bfseries,
    ndkeywords={class, export, boolean, throw, implements, import, this},
    ndkeywordstyle=\color{gray}\bfseries,
    identifierstyle=\color{black},
    sensitive=false,
    comment=[l]{//},
    morecomment=[s]{/*}{*/},
    commentstyle=\color{purple}\ttfamily,
    stringstyle=\color{red}\ttfamily,
    morestring=[b]',
    morestring=[b]"
}

\lstset{
    language=JavaScript,
    backgroundcolor=\color{lightgray},
    extendedchars=true,
    basicstyle=\footnotesize\ttfamily,
    showstringspaces=false,
    showspaces=false,
    numbers=left,
    numberstyle=\footnotesize,
    numbersep=9pt,
    tabsize=2,
    frame=single,
    breaklines=true,
    showtabs=false,
    captionpos=b
}
%
%%**************************************************************
%% Impostazioni di xcolor
%%**************************************************************
\definecolor{webgreen}{rgb}{0,.5,0}
\definecolor{webbrown}{rgb}{.6,0,0}


%**************************************************************
% Altro
%**************************************************************

%\newcommand{\omissis}{[\dots\negthinspace]} % produce [...]
%
%% eccezioni all'algoritmo di sillabazione
%\hyphenation
%{
%    ma-cro-istru-zio-ne
%    gi-ral-din
%}
%
%%\newcommand{\sectionname}{sezione}
%%\addto\captionsitalian{\renewcommand{\figurename}{Figura}
%%    \renewcommand{\tablename}{Tabella}}
%
%
%\newcommand{\glsfirstoccur}{\ap{{[g]}}}
%
\newcommand{\intro}[1]{\emph{\textsf{#1}}}

%**************************************************************
% Environment per ``rischi''
%**************************************************************
%\newcounter{riskcounter}                % define a counter
%\setcounter{riskcounter}{0}             % set the counter to some initial value
%
%%%%% Parameters
%% #1: Title
%\newenvironment{risk}[1]{
%    \refstepcounter{riskcounter}        % increment counter
%    \par \noindent                      % start new paragraph
%    \textbf{\arabic{riskcounter}. #1}   % display the title before the 
%    % content of the environment is displayed 
%}{
%    \par\medskip
%}
%
%\newcommand{\riskname}{Rischio}
%
%\newcommand{\riskdescription}[1]{\textbf{\\Descrizione:} #1.}
%
%\newcommand{\risksolution}[1]{\textbf{\\Soluzione:} #1.}
%
%%**************************************************************
%% Environment per ``use case''
%%**************************************************************
%\newcounter{usecasecounter}             % define a counter
%\setcounter{usecasecounter}{0}          % set the counter to some initial value
%
%%%%% Parameters
%% #1: ID
%% #2: Nome
%\newenvironment{usecase}[2]{
%    \renewcommand{\theusecasecounter}{\usecasename #1}  % this is where the display of 
%    % the counter is overwritten/modified
%    \refstepcounter{usecasecounter}             % increment counter
%    \vspace{10pt}
%    \begin{mdframed}[style=uc]
%        \par \noindent		                        % start new paragraph
%        {\large \textbf{\usecasename #1: #2}}       % display the title before the 
%        % content of the environment is displayed 
%        \medskip
%        }{
%        \medskip
%    \end{mdframed}
%}
%
%
%\newcommand{\usecasename}{UC-}
%
%\newcommand{\usecaseactors}[1]{\textbf{\\Attori Principali:} #1. \vspace{4pt}}
%\newcommand{\usecasepre}[1]{\textbf{\\Precondizioni:} #1. \vspace{4pt}}
%\newcommand{\usecasedesc}[1]{\textbf{\\Descrizione:} #1. \vspace{4pt}}
%\newcommand{\usecasepost}[1]{\textbf{\\Postcondizioni:} #1. \vspace{4pt}}
%\newcommand{\usecasealt}[1]{\textbf{\\Scenari Alternativi:} #1. \vspace{4pt}}

%**************************************************************
% Environment per ``namespace description''
%**************************************************************

%\newenvironment{namespacedesc}{
%    \vspace{10pt}
%    \par \noindent                              % start new paragraph
%    \begin{description}
%        }{
%    \end{description}
%    \medskip
%}
%
%\newcommand{\classdesc}[2]{\item[\textbf{#1:}] #2}


%**************************************************************
% Colori per shaded (blocchi di codice)
%**************************************************************


%\newcommand{\VerbBar}{|}
%\newcommand{\VERB}{\Verb[commandchars=\\\{\}]}
%\DefineVerbatimEnvironment{Highlighting}{Verbatim}{commandchars=\\\{\}}
%% Add ',fontsize=\small' for more characters per line
%\newenvironment{Shaded}{}{}
%\newcommand{\AlertTok}[1]{\textcolor[rgb]{1.00,0.00,0.00}{\textbf{#1}}}
%\newcommand{\AnnotationTok}[1]{\textcolor[rgb]{0.38,0.63,0.69}{\textbf{\textit{#1}}}}
%\newcommand{\AttributeTok}[1]{\textcolor[rgb]{0.49,0.56,0.16}{#1}}
%\newcommand{\BaseNTok}[1]{\textcolor[rgb]{0.25,0.63,0.44}{#1}}
%\newcommand{\BuiltInTok}[1]{#1}
%\newcommand{\CharTok}[1]{\textcolor[rgb]{0.25,0.44,0.63}{#1}}
%\newcommand{\CommentTok}[1]{\textcolor[rgb]{0.38,0.63,0.69}{\textit{#1}}}
%\newcommand{\CommentVarTok}[1]{\textcolor[rgb]{0.38,0.63,0.69}{\textbf{\textit{#1}}}}
%\newcommand{\ConstantTok}[1]{\textcolor[rgb]{0.53,0.00,0.00}{#1}}
%\newcommand{\ControlFlowTok}[1]{\textcolor[rgb]{0.00,0.44,0.13}{\textbf{#1}}}
%\newcommand{\DataTypeTok}[1]{\textcolor[rgb]{0.56,0.13,0.00}{#1}}
%\newcommand{\DecValTok}[1]{\textcolor[rgb]{0.25,0.63,0.44}{#1}}
%\newcommand{\DocumentationTok}[1]{\textcolor[rgb]{0.73,0.13,0.13}{\textit{#1}}}
%\newcommand{\ErrorTok}[1]{\textcolor[rgb]{1.00,0.00,0.00}{\textbf{#1}}}
%\newcommand{\ExtensionTok}[1]{#1}
%\newcommand{\FloatTok}[1]{\textcolor[rgb]{0.25,0.63,0.44}{#1}}
%\newcommand{\FunctionTok}[1]{\textcolor[rgb]{0.02,0.16,0.49}{#1}}
%\newcommand{\ImportTok}[1]{#1}
%\newcommand{\InformationTok}[1]{\textcolor[rgb]{0.38,0.63,0.69}{\textbf{\textit{#1}}}}
%\newcommand{\KeywordTok}[1]{\textcolor[rgb]{0.00,0.44,0.13}{\textbf{#1}}}
%\newcommand{\NormalTok}[1]{#1}
%\newcommand{\OperatorTok}[1]{\textcolor[rgb]{0.40,0.40,0.40}{#1}}
%\newcommand{\OtherTok}[1]{\textcolor[rgb]{0.00,0.44,0.13}{#1}}
%\newcommand{\PreprocessorTok}[1]{\textcolor[rgb]{0.74,0.48,0.00}{#1}}
%\newcommand{\RegionMarkerTok}[1]{#1}
%\newcommand{\SpecialCharTok}[1]{\textcolor[rgb]{0.25,0.44,0.63}{#1}}
%\newcommand{\SpecialStringTok}[1]{\textcolor[rgb]{0.73,0.40,0.53}{#1}}
%\newcommand{\StringTok}[1]{\textcolor[rgb]{0.25,0.44,0.63}{#1}}
%\newcommand{\VariableTok}[1]{\textcolor[rgb]{0.10,0.09,0.49}{#1}}
%\newcommand{\VerbatimStringTok}[1]{\textcolor[rgb]{0.25,0.44,0.63}{#1}}
%\newcommand{\WarningTok}[1]{\textcolor[rgb]{0.38,0.63,0.69}{\textbf{\textit{#1}}}}
%
%% mdframed properties
%
%%\mdfsetup{% specificare proprietà di mdframed
%
%%}
%
%\mdfdefinestyle{code}{%
%    middlelinewidth=0.3pt,
%    roundcorner=5pt,
%    innertopmargin=0.2cm,
%    innerbottommargin=0.2cm,
%    nobreak=true}
%
%\mdfdefinestyle{folder}{%
%    middlelinewidth=0.3pt,
%    roundcorner=5pt}
%
%\mdfdefinestyle{uc}{%
%    middlelinewidth=0.3pt,
%    innertopmargin=0.3cm,
%    innerbottommargin=0.1cm,
%    skipabove=0cm,
%    skipbelow=0cm,
%    nobreak=true}

% float env for code

%\trivfloat{floatcode}
%\DeclareCaptionType{floatcode}
%
%\newenvironment{code}[1][ ]
%{
%    \begin{floatcode}[#1]
%        \begin{mdframed}[style=code]
%            }
%            {
%        \end{mdframed}
%    \end{floatcode}
%}

%\captionsetup[floatcode]{name=Codice}	% nome di env 'code' %textfont=it
%\renewcommand{\listfloatcodename}{List of code blocks}
%\newlistof{code}{loc}{\listcodename}

%dirtree

%\renewcommand*\DTstylecomment{\rmfamily\color{myblue}\emph}

% Table column separation factor
\setlength{\tabcolsep}{1em}


% PYTHON

% Default fixed font does not support bold face
\DeclareFixedFont{\ttb}{T1}{txtt}{bx}{n}{10} % for bold
\DeclareFixedFont{\ttm}{T1}{txtt}{m}{n}{10}  % for normal

% Custom colors
\usepackage{color}
\definecolor{deepblue}{rgb}{0,0,0.5}
\definecolor{deepred}{rgb}{0.6,0,0}
\definecolor{deepgreen}{rgb}{0,0.5,0}

% Python style for highlighting
\newcommand\pythonstyle{\lstset{
        language=Python,
        basicstyle=\normalfont\ttfamily,
        otherkeywords={self, None},             % Add keywords here
        keywordstyle=\bfseries\color{deepblue},
        numbersep=8pt,
        emph={TransformData1, TransformData2, OtherStuff, __init__},          % Custom highlighting
        emphstyle=\normalfont\ttfamily\bfseries\color{deepred},    % Custom highlighting style
        stringstyle=\color{deepgreen},
        frame={},                         % Any extra options here
        showstringspaces=false,           % 
        backgroundcolor=\color{white}
    }}

% Python environment
\lstnewenvironment{python}[1][]
{
    \pythonstyle
    \lstset{#1}
}
{}

% JSON

\colorlet{punct}{red!60!black}
\definecolor{delim}{RGB}{20,105,176}
\colorlet{numb}{magenta!60!black}

\lstdefinelanguage{json}{
	basicstyle=\normalfont\ttfamily,
	numbers=left,
	numberstyle=\scriptsize,
	stepnumber=1,
	numbersep=8pt,
	showstringspaces=false,
	breaklines=true,
	frame=lines,
	backgroundcolor=\color{white},
	literate=
	*{0}{{{\color{numb}0}}}{1}
	{1}{{{\color{numb}1}}}{1}
	{2}{{{\color{numb}2}}}{1}
	{3}{{{\color{numb}3}}}{1}
	{4}{{{\color{numb}4}}}{1}
	{5}{{{\color{numb}5}}}{1}
	{6}{{{\color{numb}6}}}{1}
	{7}{{{\color{numb}7}}}{1}
	{8}{{{\color{numb}8}}}{1}
	{9}{{{\color{numb}9}}}{1}
	{:}{{{\color{punct}{:}}}}{1}
	{,}{{{\color{punct}{,}}}}{1}
	{\{}{{{\color{delim}{\{}}}}{1}
	{\}}{{{\color{delim}{\}}}}}{1}
	{[}{{{\color{delim}{[}}}}{1}
	{]}{{{\color{delim}{]}}}}{1},
}

%\setstretch{1.15}

% makecell package options
%\renewcommand{\cellalign}{cl}
%\renewcommand\theadalign{bc}
%\renewcommand\theadfont{\bfseries}
%\renewcommand\theadgape{\Gape[4pt]}
%\renewcommand\cellgape{\Gape[4pt]}

%\newcommand{\mycell}[2][c]{%
%    \begin{tabular}[#1]{@{}c@{}}#2\end{tabular}}